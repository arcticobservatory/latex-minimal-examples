\documentclass[
    %final
]{article}

\usepackage{ifdraft}
\usepackage{verbatim}
\usepackage{xcolor}

%======================================================================
% Scratch text
%
% This is an environment that greys out the text it contains, or if the final
% option is set, it will discard the contents altogether.
%
% Useful for text that is in the process of being rewritten.
%
\newenvironment{scratch}[1][]%
{%
    \ifoptionfinal{%
        \comment%               % discard contents
    }{%
        \begingroup%            % start a group
        \color{black!030}%      % grey out text
        \marginpar{%
            % If the scratch environment is used inline in a paragraph, then
            % then marginpar's color is not affected by the previous color
            % statement. So to be safe, change it explicitly here as well.
            \color{black!030}%
            \small%
            #1%
        }%
    }%
}
{%
    \ifoptionfinal{%
        \endcomment%
    }{%
        \endgroup%
    }%
}

\begin{document}


\section{Normal text}

Here is some text that is part of the document.

\begin{scratch}[An optional explanation can go here]
This is long, multi-paragraph text that is slated to be deleted, but
you want to keep it handy to mine for snippets and ideas.

This grey text will all disappear if the \texttt{final} option is added to the
document class.

The text in the margin is an optional argument where you can leave a comment to
explain why the text is deleted.
\end{scratch}

Here is some more text that will remain in the document.

\begin{scratch}
\section{A whole scratch section}
It is possible to remove whole sections or pages,
anything that will fit in an environment.
\end{scratch}


\section{Normal text after scratch environment}

And here is some more text that is part of the document.


It is also possible to
put some scratch text
\begin{scratch}[In-paragraph scratch text]such as this text here\end{scratch}
right in the middle of a paragraph.

\end{document}
